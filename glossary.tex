%!TEX root = main.tex
% Glossary + Acronym list
% Notice: \n{} is the remainder of an overengineered glossary / acronym solution. A lot can be simplified in the code below

% Useful commands:
% \glsresetall
% \glsentrylong
% \gls{api} - \glspl{api}


% Use this commands to print the glossary
% \glossarystyle{listhypergroup}
% \printglossary[type=main]

% Compile with using
%  makeglossaries main


% https://tex.stackexchange.com/questions/170725/glossaries-package-no-long-form-the-first-time-using-an-acronym
%If there is an unusual acronym (say XYZ) that does require the long form on its first use, then add the line
%\glslocalreset{GNU}


\usepackage{makeidx}
\makeindex


\makeatletter
\def\ifEmpty#1{\def\@temp{#1}\ifx\@temp\@empty}
\makeatother


\newcommand{\intronpl}[2][]{\emph{\glspl{glossary-{#1}}}}
\newcommand{\intron}[2][]{\intro[#1]{#2}}
\newcommand{\intro}[2][]{%
\ifEmpty{#1}%
\label{#2}%
\ifglsentryexists{glossary-{#2}}{\emph{\gls{glossary-{#2}}}}{%
\ifglsentryexists{#2}{\emph{\gls{#2}}}{%
\emph{#2}}}%
\else%
\label{#1}%
\ifglsentryexists{glossary-{#1}}{\emph{\gls{glossary-{#1}}}}{%
\ifglsentryexists{#1}{\emph{\gls{#1}}}{%
\emph{#2}}}%
\fi}

%\newcommand{\intro}[2][]{\emph{{#2}}\label{#2}\ifEmpty{#1}\index{{#2}}\else\index{#1!{{#2}}}%\index{{#2}|seealso{#1}}\fi}

\newcommand{\introgls}[2][]{\glsreset{#2}\emph{\gls{#2}}\label{#2}\ifEmpty{#1}\index{\gls{#2}}\else\index{#1!{\gls{#2}}}\index{\gls{#2}|seealso{#1}}\fi}

\newcommand{\cf}[1]{%
%(cf.
%"\nameref{#1}",
(\Vref{#1})%
}
\newcommand{\cfn}[1]{(\Vref{#1})}
\newcommand{\cfnn}[1]{\nameref{#1} (\Vref{#1})}







\usepackage[acronym,nopostdot]{glossaries}
\usepackage{etoolbox}
\makeglossaries

% \myacronym[<Optional Special Short Form>{<Referenced Label>}{<Long Form>}]
\newcommand{\myacronym}[3][]{
	\newglossaryentry{#2}
	{
	    name={\ifstrequal{#1}{}{#2}{#1}},
	    long={#3},
	    description={#3},
	    sort={#2},
	    first={\glsentrylong{#2} (\glsentryname{#2})},
	    firstplural={\glsentrylong{#2}s (\glsentryname{#2}s)},
	    type=\acronymtype
	 }
}

\usepackage{xparse}

% \myacronym[<Optional Long Form>][<Optional Special Short Form>]{<Key>}{<Description>}]
\DeclareDocumentCommand\myglossary{O{} O{} m m}{%
	\ifstrequal{#4}{}{}{
		\newglossaryentry{glossary-{#3}}{
		   long={\ifstrequal{#1}{}{#3}{#1}},
		   name={\ifstrequal{#2}{}{#3}{#2}},
		   description={\ifstrequal{#1}{}{}{(#1) }#4},
		   sort={#3},
	 	   first={\glsentrylong{glossary-{#3}}\ifstrequal{#1}{}{}{ (\glsentryname{glossary-{#3}})}}
		}
	}
	\ifstrequal{#1}{}{}{
		\newglossaryentry{#3}{
			type=\acronymtype,
			name={\ifstrequal{#2}{}{#3}{#2}},
		    long={#1},
		    description={#1},
		    sort={#3},
		    first=\glsentrylong{#3} (\glsentryname{#3}),
		    firstplural={\glsentrylong{#3}s (\glsentryname{#3}s)}
		 }
	}
}

% \n{<label>} references glossary entries and acronyms
\newcommand{\n}[1]{\ifglsentryexists{glossary-{#1}}{\gls{glossary-{#1}}}{\gls{#1}}\ifglsentryexists{#1}{\glsadd{#1}}{\glsadd{glossary-{#1}}}}
\newcommand{\npl}[1]{\ifglsentryexists{glossary-{#1}}{\glspl{glossary-{#1}}}{\glspl{#1}}\ifglsentryexists{#1}{\glsadd{#1}}{\glsadd{glossary-{#1}}}}

%\myacronym{WWW}{World Wide Web}
\newglossaryentry{WWW}{
			type=\acronymtype,
			name={Web},
		    long={World Wide Web},
		    description={World Wide Web},
		    sort={World Wide Web},
		    first={World Wide Web (WWW, \textit{Web})},
		 }

% \in{<label>} intro variant of \n
%\newcommand{\intron}[1]{}
%\ifglsentryexists{glossary-{#1}}{\glsreset{glossary-{#1}}}{\glsreset{#1}}\emph{\n{#1}}\label{#1}}

\newglossaryentry{GLAM}{
			type=\acronymtype,
			name={GLAM},
		    long={Gallery, Library, Archive, and Museum},
		    sort={GLAM},
		   	description={Gallery, Library, Archive, and Museum},
			first={\glsentrylong{GLAM} (\glsentryname{GLAM})},
		    firstplural={galleries, libraries, archives and museums (GLAMs)},
		    }

\newglossaryentry{ACID}{
	type=\acronymtype,
	name={ACID},
	sort={ACID},
	description={Atomicity, Consistency, Isolation, Durability},
	first={atomicity, consistency, isolation, durability (\glsentryname{ACID})}
}


\myacronym[GZIP]{GZIP}{Gnu ZIP}
\myacronym[LDBC SPB]{LDBCSPB}{Linked Data Benchmark Council Semantic Publishing Benchmark}
\myacronym[SP\textsuperscript{2}Bench]{SP2Bench}{{SPARQL} Performance Benchmark}
\myacronym[\ensuremath{p}]{p}{Precision}
\myacronym{AJAX}{Asynchronous JavaScript and XML}
\myacronym{API}{Application Programming Interface}{}
\myacronym{BBC}{British Broadcasting Corporation}
\myacronym{BSBM}{Berlin \gls{SPARQL} Benchmark}
\myacronym{CEP}{Complex Event Processing}
\myacronym{CERN}{European Organization for Nuclear Research}
\myacronym{CFL}{Comprehensive, Functional, Layered Architecture}
\myacronym{CH}{Cultural Heritage}
\myacronym{CI}{Continuous Integration}
\myacronym{CKAN}{Comprehensive Knowledge Archive Network}
\myacronym{CMS}{Content Management System}
\myacronym{CRUD}{Create, Read, Update, Delete}
\myacronym{CSS}{Cascading Style Sheet}
\myacronym{CSV}{Comma-separated values}
\myacronym{DBMS}{Database Management System}
\myacronym{DBPSB}{DBpedia \gls{SPARQL} Benchmark}
\myacronym{DSL}{Domain Specific Language}
\myacronym{ETL}{Extraction, Transform, Load}
\myacronym{GND}{Gemeinsame Normdatei \textit{(Integrated Authority File)}}
\myacronym{GNU}{GNU is Not Unix}
\myacronym{GTFS}{General Transit Feed Specification}
\myacronym{GUI}{Graphical User Interface}
\myacronym{HTML}{Hypertext Markup Language}
\myacronym{HTTP}{Hypertext Transfer Protocol}
\myacronym{HeBIS}{Hessisches Bibliotheks- und Informationssystem \textit{(Library and Information System of Hesse)}}
\myacronym{IETF}{Internet Engineering Task Force}
\myacronym{IMS}{Identity Mapping Service}
\myacronym{IRS}{Identity Resolution Service}
\myacronym{IR}{Information Retrieval}
\myacronym{JDBC}{Java Database Connectivity}
\myacronym{JSON}{JavaScript Object Notation}
\myacronym{JVM}{{Java} Virtual Machine}
\myacronym{KDD}{Knowledge Discovery in Databases}
\myacronym{LLOD}{Linguistic Linked and Open Data}
\myacronym{LMF}{Linked Media Framework}
\myacronym{LUBM}{Lehigh University Benchmark}
\myacronym{LarKC}{Large Knowledge Collider}
\myacronym{LoC}{Lines of Code}
\myacronym{MARC}{Machine-Readable Catalog}
\myacronym{MODS}{Metadata Object Description Schema}
\myacronym{MT}{Machine Translation}
\myacronym{NER}{Named Entity Recognition}
\myacronym{NGO}{Non Governmental Organisation}
\myacronym{NLP}{Natural Language Processing}
\myacronym{NoSQL}{Not Only SQL}
\myacronym{OAI-PMH}{Open Archives Initiative Protocol for Metadata Harvesting}
\myacronym{ODBC}{Open Database Connectivity}
\myacronym{OKFN}{Open Knowledge Foundation}
\myacronym{OLAP}{Online Analytical Processing}
\myacronym{OS}{Operating System}
\myacronym{OWL}{Web Ontology Language}
\myacronym{PCDHN}{Pan-Canadian Documentary Heritage Network}
\myacronym{PDF}{Portable Document Format}
\myacronym{PHP}{PHP Hypertext Preprocessor}
\myacronym{RDBMS}{Relational Database Management System}
\myacronym{RDFS}{\gls{RDF} Schema}
\myacronym{RDF}{Resource Description Framework}
\myacronym{REST}{Representational State Transfer}
\myacronym{RFC}{Request for Comments}
\myacronym{RSS}{Really Simple Syndication}
\myacronym{SDF}{Simple Data Format}
\myacronym{SPARQL}{SPARQL Query Language}
\myacronym{SQL}{Structured Query Language}
\myacronym{SRU}{Search/Retrieval via URL}
\myacronym{SVN}{Subversion}
\myacronym{TSV}{Tabular-separated Values}
\myacronym{UI}{User Interface}
\myacronym{ULB}{University and State Library Darmstadt}
\myacronym{UML}{Unified Modeling Language}
\myacronym{URI}{Uniform Resource Identifier}
\myacronym{URL}{Uniform Resource Locator}
\myacronym{UX}{User Experience}
\myacronym{VIAF}{Virtual International Authority File}
\myacronym{W3C}{World Wide Web Consortium}
\myacronym{WHATWG}{Web Hypertext Application Technology Working Group}
\myacronym{WSDL}{Web Services Description Language}
\myacronym{XML}{Extensible Markup Language}